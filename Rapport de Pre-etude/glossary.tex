\newglossaryentry{portefeuille}{name=portefeuille, description={Un portefeuille est une composition de plusieurs actifs financiers (actions, obligations, matières premières, etc)}}

\newglossaryentry{produit_derive}{name={produit dérivé}, description={Il s'agit d'un contrat entre deux parties, un acheteur et un vendeur, qui fixe des flux financiers futurs fondés sur ceux d'un actif sous-jacent, réel ou théorique, généralement financier.}}

\newglossaryentry{backtesting}{name=backtesting, description={
Le backtesting ou test rétro-actif de validité consiste à tester la pertinence d'une modélisation ou d'une stratégie en s'appuyant sur un large ensemble de données historiques réelles.}} 

\newglossaryentry{drawdown}{name=Drawdown, description={Le maximum Drawdown signifie la perte maximale historique qu'aurait subi un investisseur malchanceux s'il avait acheté au plus haut et revendu au plus bas durant un temps déterminé.}} 

\newglossaryentry{Ruban}{name=ruban, description={
Le ruban est une interface utilisateur graphique basée sur le principe des widgets, composée d'un bandeau qui regroupe toutes les fonctions du logiciel. L'utilisateur peut trouver en un seul endroit toutes les fonctionnalités, avec des rubans adaptés au contexte des données.}}

\newglossaryentry{Scénario de stress}{name={Scénario de stress}, description={
Un scénario de stress est une technique de simulation utilisée sur un portefeuille afin de déterminer sa réaction à différentes situations financières, en particulier une crise.}}