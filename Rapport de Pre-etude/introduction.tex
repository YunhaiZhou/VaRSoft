\chapter{Introduction} 
	Malgré l'augmentation de la volatilité des marchés financiers, le développement de \gls{produit_derive} et surtout une série de faillites et de krachs boursiers, il faudra attendre 1996 et les accords de Bâle I, puis Bâle II en 2001, pour que des dispositions bancaires internationales obligent les banques à se prémunir des pertes imprévues. 

De plus, bien qu'aucun outil de mesure de risque n'ait été imposé par les accords, les institutions financières se mirent à développer leurs outils. En effet, des pénalités peuvent être appliquées aux banques ayant une mauvaise gestion des risques.

Bien que les accords de Bâle concernent plus les risques systémiques du secteur bancaire que la finance de marché à proprement dit, ce premier accord de Bâle sera le point de départ de la réalisation et l'utilisation massive d'un nouvel indicateur, considéré très rapidement par les institutions financières comme un standard dans l'évaluation des risques financiers, la Value-At-Risk (VaR).

La VaR dans les accords de Bâle est plutôt utilisée pour déterminer une limite de fonds propres que les banques doivent pouvoir conserver afin de ne pas faire faillite. En finance de marché, la VaR est utilisée pour mesurer un risque de marché, se couvrir contre ce risque et, in fine, optimiser le rendement d'un portefeuille de titres.

Démocratisée par la banque JP Morgan en 1993 via la méthode RiskMetrics, la notion de VaR a été utilisée pour la première fois dans les années 1980 par la banque Bankers Trust, sur les marchés financiers américains. Jusqu'alors il était coutume d'utiliser la variance ou le \gls{drawdown}, qui sont d'autres mesures de risque de marché.

L'objectif de ce projet est de développer une application permettant d'utiliser des outils statistiques pour le calcul de risques en finance. Destiné aux acteurs de la finance, ce logiciel devra présenter une interface simple à utiliser. L'utilisateur pourra importer des données brutes telles que les cours des actions du CAC 40, gérer différents portefeuilles et comparer les résultats des différentes méthodes statistiques, ainsi que calculer en particulier la VaR. Pour permettre d'évaluer quels sont les modèles statistiques les plus adaptés aux données d'entrée, un ensemble de tests et de contrôles seront possibles sur les résultats obtenus selon les différents algorithmes de calcul.

Le développement de ce logiciel tiendra compte de l'existant. Plusieurs logiciels permettent déjà de calculer la VaR et d'utiliser différentes méthodes statistiques pour calculer des risques financiers, que ce soient des modules à intégrer à Excel ou des logiciels spécialisés. 
L'avantage que présentera notre programme se situe principalement dans les fonctionnalités de \gls{backtesting} et de tests de comparaison sur les différentes méthodes statistiques. De plus, l'interface sera plus ergonomique, plus simple du fait du nombre de fonctionnalités (inférieur à celui d'Excel par exemple). Le logiciel produit sera libre et implémentera des fonctionnalités inédites ou normalement payantes.

Ce rapport de pré-étude vient donner les conclusions de la phase d'analyse du projet. Dans un premier temps, nous nous attacherons à exposer le contexte mathématique dans lequel évolue le projet. Cela inclut les séries temporelles, leurs fondements mathématiques ainsi que les différentes méthodes qui seront utilisées dans le projet.

La seconde partie du rapport tiendra lieu de cahier des charges du logiciel. Nous y détaillerons notamment les différentes fonctionnalités souhaitées. Nous évoquerons la manière dont nous voyons leur intégration dans le logiciel.

Dans la dernière partie du rapport, nous présenterons les différents outils et technologies utilisés, puis nous détaillerons le planning prévisionnel ainsi que la répartition des tâches envisagés. Pour finir, nous nous interrogerons sur la licence sous laquelle nous placerons notre logiciel, ce qui déterminera notamment les conditions de sa réutilisation pour de futurs développements.
