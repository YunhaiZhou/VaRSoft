\documentclass[a4paper]{report}
\usepackage[T1]{fontenc}
\usepackage[french]{babel}
\usepackage[utf8x]{inputenc}
\usepackage{amsmath}
\usepackage{eurosym}
\usepackage{float}
\usepackage{graphicx}
\usepackage[colorinlistoftodos]{todonotes}
\usepackage{placeins}
\usepackage{verbatim}
\usepackage{fmtcount}
\usepackage{array}
\usepackage[toc,page]{appendix} 
\renewcommand{\arraystretch}{1.25}

\usepackage{tabularx}
\newcounter{cptspec}
\setcounter{cptspec}{0}
 
\title{INSA de Rennes \\ Quatrième année Informatique \\ \bigskip \hrule \bigskip Rapport bilan de planification \\ \bigskip Projet VaR \bigskip \hrule}

\author{Benjamin \bsc{Bouguet} - Paul \bsc{Chaignon} \\Eric \bsc{Chauty} - Ulysse \bsc{Goarant} \\ ~~\\
Hamdi \bsc{Raissi} - Ivan \bsc{Le Plumey} \\ Quentin \bsc{Giai Gianetto}}


\date{Mai 2014}

\begin{document}
\maketitle

\thispagestyle{empty}
\newpage

~~
\thispagestyle{empty}
\newpage


\tableofcontents
\newpage


\chapter{Introduction}

En décembre 2013, nous avions établi une estimation du temps de travail à réaliser sur chaque partie du logiciel, que ce soit du développement ou des tests, chaque point avait été détaillé.
Durant les quelques mois qui ont suivi, un décompte du temps passé sur chaque étape du développement ou des tests a été effectué.

Aujourd'hui nous sommes en mesure d'effectuer un bilan de notre planification initiale.
Nous reprendrons les fonctionnalités précédemment détaillées dans le rapport de planification, en passant en revue sur les éventuelles différences observées entre la durée estimée et la durée réelle.

Nous avions également effectué des prévisions de temps de travail sur la conception, la page HTML, la réalisation du rapport final et la préparation à la soutenance.
Nous n'en parlerons pas dans ce document car ces dernières sont secondaires dans les enseignements à retenir des erreurs de planification.
En effet, ces étapes se sont toutes bien déroulées ou ne sont pas encore terminées.


\chapter{Développement}
\noindent Le développement est composé de sept grandes parties :
\begin{itemize}
  \item les données
  \item les calculs
  \item le backtesting
  \item l'IHM
  \item les graphiques
  \item les rapports
  \item les sessions
\end{itemize}

Nous dresserons dans cet ordre le bilan des différentes phases du développement.
Comme prévu, nous avons eu l'avantage que beaucoup de nos tâches ont été indépendantes les unes des autres.
Ceci nous a permis d'avoir une plus grande flexibilité dans la répartition des tâches et leur réalisation.
Cependant, nous avons eu un surplus de travail (conflits) lors de la fusion des différentes branches de développement.

\section{Gestion des données}
Le calcul de la Value-At-Risk s'effectuant sur des données financières, il est nécessaire de commencer par implémenter la gestion des données.

\subsection{Actifs}
Les actifs constituent les éléments de base manipulés dans le logiciel.
La première étape a été l'implémentation du modèle obtenu lors de la phase de conception.
C'est à dire créer la classe et les méthodes associées.

Notre logiciel est capable d'importer des données provenant de Yahoo Finance aux formats CSV et texte.
L'importation est consitutée d'un ensemble de traitements sur le fichier des actifs et résulte en la création d'un nouveau fichier ne contenant que les données sélectionées.

\begin{table}[H]
\centering
  \begin{tabularx}{0.8\textwidth}{| X | c | c |}
    \hline
	Tâches & Durée prévue & Durée réelle\\
    \hline
    Modélisation des actifs & 10h & 11h\\
    \hline
  \end{tabularx}
  \caption{Durées des tâches liées aux actifs}
\end{table}

Nous avons passé presque exactement le temps prévu sur l'importation des données.
Cependant nous aurions pu être plus rapide.
En effet, nous avons réalisé cette fonctionnalité au cours de trois versions; les deux dernières constituant des améliroations de la première.
Avec plus de discussions entre nous avant le développement, nous aurions peut-être pu nous passer des deux dernières versions.

\subsection{Portefeuilles}
Les portefeuilles sont constitués d'une composition d'actifs.
Il était donc nécessaire d'avoir préalablement implémenté la gestion des actifs avant d'envisager l'implémentation de la gestion des portefeuilles.

\begin{table}[H]
\centering
  \begin{tabularx}{0.8\textwidth}{| X | c | c |}
    \hline
	Tâches & Durée prévue & Durée réelle \\
    \hline
     Modélisation des portefeuilles &  20h & 19h30\\
    \hline
  \end{tabularx}
  \caption{Durées des tâches liées aux portefeuilles}
\end{table}

Aucun souci particulier n'ayant été rencontré durant la réalisation des différentes classes nécessaires à la gestion des portfeuilles, nous avons presque égalé le temps prévu, notre estimation était donc bonne.

\section{Importation et exportation}

\begin{table}[H]
\centering
  \begin{tabularx}{0.8\textwidth}{| X | c | c |}
    \hline
	Tâches & Durée prévue & Durée réelle \\
    \hline
     Importation &  35h & 4h30\\
     Exportation &  20h & 9h\\
    \hline
  \end{tabularx}
  \caption{Durées des tâches liées à l'importation/exportation}
\end{table}

Comme expliqué dans le rapport final, nous avons modifié la spécification associé à l'importation et l'exportation des données.
Les données sont maintenant toutes importées et exportées en une seule fois sans que l'utilisateur puisse choisir lesquels.

En conséquence, ces fonctionnalités ont été bien plus rapide que prévu à développer.

Cependant nous avions prévu de passer plus de temps sur l'importation que l'exportation et, même si les deux ont été développées en même temps, l'exportation a été difficile à mettre en place.
Cela est dû à la nécessité d'utiliser une librairie externe pour écrire l'archive.
Nous avons donc installé et utilisé la librairie QuaZip.


\section{Calculs}

Les fonctionnalités de calcul contiennent principalement le calcul de la Value-at-Risk selon les différentes méthodes.
Pour calculer cette dernière, nous avons utilisé le logiciel de statistiques R.
Dans un premier temps, nous avons cherché à interfacer notre logiciel avec R en utilisant RInside, conformément aux spécifications.
Cependant, cela s'est révélé être très compliqué \footnote{Le rapport final explique ce problème plus en détails et, rapidement, les démarches que nous avons entrepris pour essayer de le résoudre.} malgré toutes nos recherches et essais.
Nous avons dû trouver une autre solution.

Les tests de corrélation nous ont été fournis sous forme de scripts R par notre encadrant.
Il nous a d'abord fallu les tester (ce qui correspond à la tâche de test des scripts externes) puis les appeler via R.
Finalement, nous avons passé moins de temps que prévu sur l'interfaçage avec R.
Notre solution risque cependant d'être moins stable dans le temps que si nous avions utilisé une librairie externe comme RInside.

Enfin, les calculs statistiques se résument à de simples appels à des fonctions fournies par R.
C'est pourquoi cette tâche est relativement courte.

\begin{table}[H]
\centering
  \begin{tabularx}{0.8\textwidth}{| X | c | c |}
    \hline
	Tâches & Durée prévue & Durée réelle \\
    \hline
    Interfaçage avec R &  20h  & 14h30\\
    Modélisation GARCH &  15h & 11h30\\
    Méthode historique &  5h & 9h\\
    Méthode Riskmetrics &  10h & 2h\\
    Tests de corrélation &  10h & ?\\
    Tests scripts externes R & 5h & 4h\\
    Calculs statistiques &  5h & ?\\
    \hline
  \end{tabularx}
  \caption{Durées des tâches liées aux calculs}
\end{table}

Bien que le temps passé sur la méthode historique excède celui prévu initialement, cela se doit au fait qu'un code en commun à d'autres classes a été développé.
De plus, des dépendances avec d'autres branches ont nécessité plusieurs modifications.
En elle-même, la méthode historique a été rapide à développer.

Comme nous le pensions, la réalisation de la méthode Riskmetrics a été plus rapide, car grandement inspirée de la modélisation GARCH.
Plus de travail en préparation sur les notions mathématiques à implémenter nous a permis de développer cette fonctionnalité encore plus rapidement que prévu.

De même que précédemment, la méthode GARCH s'est révelée plus rapide à réaliser, une connaissance moins floue sur Qt et les notions mathématiques utilisées, aurait pu donner une meilleure évaluation.


\section{Backtesting}

Le développement du backtesting a nécessité préalablement le fonctionnement du calcul de la VaR selon les différentes méthodes de calcul.

\begin{table}[H]
\centering
  \begin{tabularx}{0.8\textwidth}{| X | c | c |}
    \hline
	Tâches & Durée prévue & Durée réelle \\
    \hline
    Backtesting &  30h & 3h\\
    \hline
  \end{tabularx}
  \caption{Durées des tâches liées au backtesting}
\end{table}

Cette étape a été complètement sur-évaluée.
Ceci pourrait peut être s'expliquer de nouveau par le manque de certaines notions mathématiques, mais aussi par la bonne réalisation des calculs de VaR précédemment.
Le backtesting étant totalement basé dessus, et ne fait que lancer des tests de comparaisons entre les méthodes de calcul de VaR.


\section{Rapports}

De manière générale, les rapports résument les informations générées par les autres modules du logiciel.
Il n'était cependant pas nécessaire que le module correspondant au rapport à générer soit d'abord implémenté.
Le développement en parallèle du module de la génération des rapports et des modules correspondants d'où sont extraites les données, a donc été possible.

\begin{table}[H]
\centering
  \begin{tabularx}{0.8\textwidth}{| X | c | c |}
    \hline
	Tâches & Durée prévue & Durée réelle \\
    \hline
    Rapport au format DOCX &  10h & 23h\\
    Rapport au format PDF &  10h & 0h\\
    Rapport du calcul de la VaR & 5h & 2h\\
    Rapport de statistiques générales & 10h & 2h\\
    Rapport de backtesting & 5h & 2h\\
    Rapport des tests de corrélation & 5h & 2h\\
    \hline
  \end{tabularx}
  \caption{Durées des tâches liées aux rapports}
\end{table}

Dans le rapport de spécification, nous avions prévu d'utiliser la librairie libopc pour générer les rapports en PDF ou DOCX.
Cela s'est révelé être une mauvaise idée, cela était bien trop complexe.
Nous avons opté pour l'utilisation et la réalisation d'un programme Java, auquel notre logiciel envoie un ensemble de paramètres représentant les données à ajouter dans le rapport.

Cela explique donc pourquoi nous avons dépassé le temps prévu pour la réalisation des rapports en DOCX, car ce fût les premiers réalisés.
Avec la librairie Java que nous avons utilisé, la conversion du format DOCX au format PDF est très simple.
Nous avons donc fait cela en même temps que la génération du format DOCX.

Le développement de notre propre programme Java pour faire l'interface avec la librairie de génération nous a permis de gagner du temps sur la suite de la génération.
En effet, grâce à ce programme, la génération de nouveaux rapports est très simple à mettre en place.
Elle nécessite uniquement de modifier le \textit{template} \footnote{Plus d'explications sur le fonctionnement de notre programme sont données dans le rapport final.}.


\section{Graphiques}

Les graphiques nécessitent principalement l'implémentation de la gestion des données.
L'implémentation des matrices de corrélation nécessite de plus celle des calculs correspondants.

\begin{table}[H]
\centering
  \begin{tabularx}{0.8\textwidth}{| X | c | c |}
    \hline
	Tâches & Durée prévue & Durée réelle \\
    \hline
    Matrice de corrélation &  15h & 6h\\
    Histogramme &  15h & 4h\\
    \hline
  \end{tabularx}
  \caption{Durées des tâches liées aux graphiques}
\end{table}


\section{IHM}

L'IHM contient toutes les actions que l'utilisateur peut effectuer, les formulaires que l'utilisateur devra remplir ainsi que les éléments graphiques tels que les tableaux.
La durée des tâches comprend aussi la gestion des messages d'erreurs ainsi que leur affichage dans l'IHM.

\begin{table}[H]
\centering
  \begin{tabularx}{0.8\textwidth}{| X | c | c |}
    \hline
	Tâches & Durée prévue & Durée réelle \\
    \hline
    Exporter des actifs & 5h & ?\\
    Importer des actifs & 10h & 4h\\
    Backtesting & 5h & ?\\
    Calcul de la VAR & 5h & ?\\
    Ajouter/Modifier portefeuille & 10h & 3h\\
    Gestion des sessions & 10h & ?\\
    Générer rapports & 5h & 5h\\
    Calculs statistiques & 5h & ?\\
    Exporter un portefeuille & 10h & ?\\
    Importer un portefeuille & 5h & ?\\
    Supprimer un portefeuille & 5h & ?\\
    Listing des portefeuilles & 10h & 1h30\\
    Afficher le contenu d'un portefeuille & 10h & 14h\\
    Afficher les rapports d'un portefeuille & 5h & 5h\\
    \hline
  \end{tabularx}
  \caption{Durées des tâches liées à l'IHM}
\end{table}

Pour la plupart des éléments graphiques, de simples intéractions entre les objets Qt ont été suffisantes.
C'est pour cela que le temps prévu s'est révélé trop important.

L'affichage du contenu des portefeuilles été plus compliqué.
Nous aurions pu nous limiter à une simple lecture à la base de données puis à un affichage dans un QTableWidget, mais nous avons cherché à avoir une solution plus évolutive.Nous donc sommes partis sur la réalisation d'un QAbstractItemModel, qui a demandé plus de développement que notre première solution.


\section{Session}

Cette fonctionnalité ajoute la sauvegarde dans la base de données des portefeuilles, rapports et actifs.
Les modifications sont aussi sauvegardées (changement de nom, suppression d'un rapport, changement de l'emplacement du fichier d'un rapport).
Cette tâche a nécessité que la gestion des actifs et des portefeuilles soit en partie déjà réalisée.

\begin{table}[H]
\centering
  \begin{tabularx}{0.8\textwidth}{| X | c | c |}
    \hline
	Tâches & Durée prévue & Durée réelle \\
    \hline
    Sauvegarde de la session &  10h & 6h30\\
    Restaurer la session &  10h & 10h\\
    \hline
  \end{tabularx}
  \caption{Durées des tâches liées à la session}
\end{table}

Nous avons devéloppé cette fonctionnalité vers la fin du projet.
Nous maîtrisions donc mieux Qt et la librairie pour SQLite.
Nous avons donc gagné un peu de temps sur cette tâche.


\chapter{Tests}

Comme expliqué dans le rapport final, nous avons effectué les tests au fur et à mesure des branches de développement.
L'IHM étant plus difficile à tester automatiquement, celle-ci a été testée (manuellement) en suivant des scénarios.
Nous avons établis ces scénarios en même temps que nous développions les fonctionnalités.

\begin{table}[H]
\centering
  \begin{tabularx}{0.8\textwidth}{| X | c | c |}
    \hline
	Tâches & Durée prévue & Durée réelle \\
    \hline
    Tests de la gestion des données & 20h & 35h\\
    Tests des calculs (modèle et VaR) & 20h & 6h30 + ?\\
    Tests IHM & 20h & 5h + ?\\
    \hline
  \end{tabularx}
  \caption{Durées des tâches liées aux tests}
\end{table}

Nous avons passé plus de temps que prévu sur les tests uniatires et fonctionnels automatiques.
En effet, nous avons notamment dû configurer l'outil d'intégration continue, Travis CI, qui nous permet maintenant de connaître les résultats des tests après tout changement.
Cet outil nous assure cependant une plus grande qualité logicielle et simplifie le travail pour la relecture des branches.

Deux principaux tests ont été prévus concernant la modélisation du modèle Garch. Le premier est basé sur le contrôle des données d'entrée pour comparer le résultat obtenu avec une valeur théorique pré-déterminée. Ce test est en cours et une partie de son temps a été décompté. Le deuxième test fait usage quant à lui du backtesting. Cette fonctionnalité étant sur le point d'être validée, ce test pourra être alors effectué.

Les tests de l'interface graphique et des calculs ne sont pas encore complètement finis donc nous ne pouvons qu'estimer le temps qui y sera finalement consacré.


\chapter{Diagramme de GANTT}


\chapter{Conclusion}

Le tableau ci-dessous permet de comparer la planification initiale du projet global avec le temps réellement nécessaire à sa réalisation :

\begin{table}[H]
\centering
  \begin{tabularx}{0.8\textwidth}{| X | c | c |}
    \hline
	Tâches & Durée prévue & Durée réelle \\
    \hline
    Développement & 375h & ?\\
    Tests & 60h & ?\\
    \hline
	Total & 435h & ?\\
    \hline
  \end{tabularx}
  \caption{Durées des tâches principales}
\end{table}

Dans ce document, nous avons comparé les temps prévus à la réalisation des différentes parties de notre logiciel, en expliquant dans la mesure du possible les raisons pour lesquelles nous avons observé des différences entre la durée prévue et la durée effective.
Globalement, le temps passé sur la réalisation de ce projet est inférieur au temps prévu.
Cela ne signifie pas que moins de choses ont été réalisées, mais plutôt qu'elles avaient été sur-évaluées en temps de développement ou que nous avons été plus efficace.

Cependant nous avons quand même dû réduire légèrement les possibilités de notre logiciel \footnote{Voir les changements apportés aux spécifications dans le rapport final.}.
En effet, nous avions complètement surestimé le temps que nous pouvions consacrer à ce projet chaque semaine.
Nous avions notamment sous-estimé le temps nécessaire pour les autres matières et projets à travailler au cours du second semestre.


\end{document}